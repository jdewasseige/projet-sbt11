\section{Contexte du projet}

\subsection{Client du Projet}
Pierre Carmier, le client du projet, étant un chercheur à Digiplante,
une équipe à labo du MAS. Cette équipe travaillait notamment sur le modèle
mathèmethique de croissance de plantes depuis 2002. 
Ils travaillent notatement sur le modèle LNAS qui a bien adapté
à un mono type de plante.

\subsection{Contexte}
Le problème de ressoures nous attire plus en plus d'attention depuis le 21ème siècle
surtout en argiculture. Donc il nous posse à optimiser la gestion pour l'argiculture
à réduire le volume des ressources en intrants ( eau, fertilisants, etc.)
et récolter aussi bien de produits enfin d'améliorer l'efficité
avec des critères économiques et écologiques.

D'autre part, le développement de technologie (l'imagerie satellitaire, drones etc.)
nous permet d'obtenir de plus en plus précises et nombreuses donées. 
Avec des nombreuses données, nous pouvons le résoudre avec l'informatique en
développer des modèles décisionnel, précis et pratiques. 
Tous nous rendent des résultats plus précis qu'avant.

Donc il sort le dévelop de \"smart agriculture\" dont buts est de préparer
l'agriculture de demain qui nous permet de nourrir la plante écologiquement et
économiquement. Et nous concentrons aur la modélisation de la croissance des plantes
pour l'agriculture de précision comme une partie et nous attedons les résultats 
d'un programme capable de s'adapter à des nouveaux modèles de croissance et 
aussi la pertinence des paramètres pour des utilisations pratiques. 

La plante apparait comme un exemple de systèm complexe typique concernant 
multiple échelles. Il est non-linéaire et possède beaucoup de paramètres
d'environnement et genetic et (regulation and retroaction loops?en francais??).
De plus,  le procéssus biologique est complèxe et a intersection entre eux
(http://digiplante.mas.ecp.fr). Tous sont des contraintes que 
nous devons à respecter. Mais heuresment, 
il existe déjà plusieurs modèles possibles pour certaines parts 
par example le LNAS et le plateforme de \textsc{Julia}.

Nous esperons de trouver des modèles plus flexible qui adapte mieux à la diversité
de variables et conditions que l'on observe mais aussi avec moins d'incertitude 
par rapport aux données possible. Avec des travaux 
nous pouvons bien les appliquer dans l'agriculture 
mais aussi de prédire la croissance de plante.
