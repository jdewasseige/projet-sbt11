\section{Contexte du projet}

\subsection{Client du Projet}

Pierre Carmier, le client du projet, \'etant un chercheur à Digiplante, une \'euipe à labo du MAS. Cette \'equipe travaillait notamment sur le mod\`ele math\`emethique de croissance de plantes depuis 2002. Ils travaillent notatement sur le mod\`ele LNAS qui a bien adapt\'e \`a un mono type de plante.
\subsection{Contexte}

Le probl\`eme de ressoures nous attire plus en plus d'attention depuis le 21\`eme si\`ecle surtout en argiculture. Donc il nous posse \`a optimiser la gestion pour l'argiculture \`a r\'eduire le volume des ressources en intrants ( eau, fertilisants, etc.) et r\'ecolter aussi bien de produits enfin d'am\'eliorer l'efficit\'e avec des crit\`eres \'economiques et \'ecologiques.

D'autre part, le d\'eveloppement de technologie (l'imagerie satellitaire, drones etc.) nous permet d'obtenir de plus en plus pr\'ecises et nombreuses don\'ees. Avec des nombreuses donn\'ees, nous pouvons le r\'esoudre avec l'informatique en d\'evelopper des mod\`les d\'ecisionnel, pr\'ecis et pratiques. Tous nous rendent des r\'esultats plus pr\'ecis qu'avant.

Donc il sort le d\'evelop de 'smart agriculture' dont buts est de pr\'eparer l'agriculture de demain qui nous permet de nourrir la plante \'ecologiquement et  \'economiquement. Et nous concentrons aur la mod\'elisation de la croissance des plantes pour l'agriculture de pr\'ecision comme une partie et nous attedons les r\'esultats d'un programme capable de s'adapter \`a des nouveaux mod\`eles de croissance et aussi la pertinence des param\`etres pour des utilisations pratiques. 

La plante apparait comme un exemple de syst\`em complexe typique concernant multiple \'echelles. Il est non-lin\'eaire et poss\`ede beaucoup de param\`etres d'environnement et genetic et (regulation and retroaction loops?en francais??). De plus,  le proc\'essus biologique est compl\`exe et a intersection entre eux (http://digiplante.mas.ecp.fr). Tous sont des contraintes que nous devons \`a respecter. Mais heuresment, il existe d\'ej\`a plusieurs mod\`eles possibles pour certaines parts par example le LNAS et le plateforme de Julia.

Nous esperons de trouver des mod\`eles plus flexible qui adapte mieux \`a la diversit\'e de variables et conditions que l'on observe mais aussi avec moins d'incertitude par rapport aux donn\'ees possible. Avec des travaux nous pouvons bien les appliquer dans l'agriculture mais aussi de pr\'edire la croissance de plante.