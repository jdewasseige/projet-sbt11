\section{Contexte du projet}

\subsection{Client du Projet}
Pierre Carmier, le client du projet, est chercheur au laboratoire MAS. Ce laboratoire travaille notamment sur les modèles mathématiques de croissance des plantes, en collaboration avec la Start-Up Digiplante. Ils ont ainsi travaillé sur le modèle LNAS, que nous allons utiliser dans notre projet pour modéliser la croissance du blé.
\subsection{Contexte}
Le problème des ressources énergétiques et alimentaires est un sujet crucial du  21ème siècle. Il faudra être capable de nourrir plus de 9 milliards d'humains en 2050
A l'avenir, il faudra donc optimiser l'agriculture, en réduisant le volume des ressources en intrants ( eau, fertilisants, etc.) tout en améliorant les rendements.

D'autre part, le développement de technologies (GPS, drones, etc.) nous permet d'obtenir des données de plus en plus précises. 
Cela permet d'envisager d'augmenter les rendements agricoles, tout en réduisant la consommation de ressources, en exploitant ces données grâce à des modèles décisionnels, précis et pratiques. 

On assiste donc au développement d'une agriculture intelligente dont le but est de préparer
l'agriculture de demain.

Pour notre part, nous allons nous intéresser à la modélisation de la croissance des plantes et à la pertinence des mesures à réaliser.
