\section{Organisation du groupe}
Nous présenterons dans cette section l'organisation générale du groupe.
On abordera d'abord les outils que nous avons utilisés pour partager notre travail
le plus efficacement possible.
On présentera ensuite notre gestion de la bibliographie.
On détaillera finalement l'organisation interne du groupe,
plus précisemment le système mis en place pour optimiser la communication
ainsi que la manière dont les tâches ont été réparties pendant ce premier semestre.

\subsubsection*{Outils de travail collaboratif}
Pour des raisons pratiques et esthétiques, nous avons décider d'écrire
nos rapports en \LaTeX.
Il s'agissait donc de trouver la meilleure façon de partager le code source
et de pouvoir contrôler les changements apportés au document.
Une première idée pourrait être d'utiliser ShareLaTeX qui propose une plate-forme
de compilation en ligne ainsi qu'un système de gestion de versions
assez simple à utiliser.
Nous n'avons pas choisi cette solution notamment pour les raisons suivantes.
L'utilisateur doit être connecté dès qu'il veut travailler sur le projet,
le système de compilation est assez lent et l'utilisateur n'est pas libre
d'utiliser son éditeur de texte ou son visualisateur de \textsc{pdf} favori.

Pour palier aux problèmes décrits ci-dessus, le logiciel \texttt{git}
est une très bonne alternative.
Il permet en effet à chaque membre du groupe de travailler localement
et de ne faire un \emph{push} sur le serveur que lorsqu'on juge
que ce qu'on a fait est utile pour les autres.
L'algorithme du fusion, \emph{merge}, permet également de fusionner intelligemment
les lignes d'un fichier qui ont été modifiées par plusieurs membres.
Le dernier point à souligner est qu'il permet une gestion des branches,
particulièrement pratique lorsqu'on veut développer une partie du projet
sans risquer de créer des erreurs dans le programme principal.

Nous l'utilisons pour implémenter un programme en \textsc{Julia}
pour le client dont le code source est sur la plate-forme GitLab.
Nous avons également un dossier sur
GitHub\footnote{\url{https://github.com/jdewasseige/projet-sbt11}}
qui contient nos fichiers pour rédiger l'étude documentaire.

\subsubsection*{Gestion de la bibliographie}
Pour la gestion de la bibliographie au sein du document,
nous utilisons le package \texttt{biblatex}.
Celui-ci permet d'écrire l'ensemble de nos références dans un document \texttt{.bib}
sous la forme suivante.
\begin{verbatim}
  @online{histoire_mod_plantes,
    title = {Une histoire de la modélisation des plantes},
    author = {Philippe de Reffye and Marc Jaeger 
    and Paul-Henry Cournède},
    url = {https://interstices.info/jcms/c_38032/une-histoire-de-
    la-modelisation-des-plantes},
    year = {2009},
    month = "04",
  }
\end{verbatim}
La mise en page est alors automatique en fonction des informations fournies
et le rendu de l'exemple est présenté ci dessous.
\begin{figure}[h]
  \includegraphics[scale=0.6]{./img/rendu_elem_bib.jpg}
\end{figure}

Cela parait à priori assez lourd d'écrire soi-même toutes les informations
en suivant cette syntaxe mais il existe des logiciels comme Zotero
qui font le travail à notre place.
Les sources trouvées sur Google Scholar peuvent également être exportées
aisément au format \texttt{BibTex}.
           
\subsubsection*{Organisation de l'équipe et partage de tâches}
Il nous reste un dernier point à décrire, celui de la \emph{communication}
au sein du groupe.
Nous utilisons Slack\footnote{\url{https://slack.com/}} qui est un logiciel
de plus en plus utilisé pour les travaux de groupe ainsi que dans les start-ups.
Il permet d'éviter de devoir alterner entre plusieurs applications comme les mails,
DropBox et Twitter, puisqu'il permet d'être connecté
à celles-ci au sein de l'application.
On peut également créer plusieurs \emph{channels} pour séparer la communication
entre les différentes tâches.
Par exemple dans ce projet nous avons les \emph{channels} suivantes :
\texttt{general}, \texttt{etude-documentaire}, 
\texttt{planning} et \texttt{dev\_plate-forme}.
On trouve aussi un système d'historique et de gestion de fichiers efficace.

%Le partage de tâches est associé au planning qui se trouve
%dans l'Annexe~\ref{ann:planning} à la page~\pageref{ann:planning}.

% TODO planning en annexe
