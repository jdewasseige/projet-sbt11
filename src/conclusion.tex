\section*{Conclusion}
\addcontentsline{toc}{section}{Conclusion}

Les perspectives qu'offre l'étude de la croissance des plantes sont des plus stimulantes.
En effet, les enjeux sont cruciaux. Les plantes sont un élément clé de notre économie. A la base de nos régimes alimentaires, elles sont également source d'oxygène, d'énergie avec l'émergence des agro-carburants, et sont utilisées dans la construction, l'industrie des vêtements et même dans l'industrie pharmaceutique.



 Dans le même temps, pour répondre à une demande de plus en plus forte, l'agriculture a recours à l'irrigation intensive, aux engrais et aux pesticides pour augmenter ses rendements. Malheureusement, ces méthodes ont un impact négatif sur l'environnement et leur utilisation n'est pas viable, comme en témoigne par exemple le quasi-asséchement de la mer d'Aral à cause d'une culture intenisive du coton~\cite{aral}
 
Heureusement, la science, aidées des mathématiques et d'une collecte de données devenue omniprésente, semble plus que jamais capable de développer une agriculture viable et productive. Pour cela, elle bénéficie de plusieurs siècles d'études des plantes qui ont permis d'affiner leurs connaissances.


