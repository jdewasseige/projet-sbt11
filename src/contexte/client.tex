\subsection{Présentation du client}
Pierre Carmier, le client du projet, est chercheur au laboratoire MAS. Ce laboratoire travaille notamment sur les modèles mathématiques de croissance des plantes, en collaboration avec la Start-Up Digiplante. Ils ont ainsi travaillé sur le modèle LNAS, que nous allons utiliser dans notre projet pour modéliser la croissance du blé.
\subsection{Contexte}
Le problème des ressources \textit{énergétiques} et \textit{alimentaires} est un sujet crucial du  21ème siècle.
Il faudra être capable de nourrir plus de 9 milliards d'humains en 2050. De plus, les ressources fossiles et l'eau douce viennent à manquer dans certains régions agricoles (comme en Californie). Pourtant, l'agriculture nécessite l'apport d'eau, et est grande consommatrice d'énergie. L'agriculture est ainsi responsable de l'émission de près de 20\% des gaz à effet de serre\cite{GES}.

A l'avenir, il faudra donc optimiser l'agriculture en améliorant les rendements (pour nourrir une population plus grande), tout en réduisant le volume des ressources utilisées (eau, fertilisants, etc.).

D'autre part, le développement de technologies de plus en plus sophistiquées (GPS, drones, etc.) nous permet d'obtenir des données de plus en plus précises.
Cela permet d'envisager le développement d'une agriculture intelligente qui exploiterait ces données grâce à des modèles décisionnels, précis et pratiques et permettrait d'atteindre ce double objectif. On peut ainsi imaginer qu'un jour, un tracteur géolocalisé disperse plus ou moins de graines dans certains zones d'un même champ, afin d'obtenir de meilleurs rendements.

Pour notre part, nous allons nous intéresser à la modélisation de la croissance des plantes et à la pertinence des mesures à réaliser.
