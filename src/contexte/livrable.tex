\subsection{Objectifs poursuivis}
\subsubsection{Implémentation du modèle LNAS sous \textsc{Julia}}
Nous allons travailler sur le Modèle LNAS, adapté à la modélisation du blé. 
Le client a déjà implémenté ce modèle en C++. 
Nous allons pour notre part l'implémenter sous Julia.

Il y a environ 30 paramètres dans le modèle. Une fois le modèle implémenté, nous essaierons donc d'estimer ces paramètres par calibration, grâce aux outils fournis par la statistique.

De plus, les mesures étant difficiles à réaliser et ayant un coût, on ne peut pas mesurer toutes les variables tout le temps. A terme, le but serait "d'obtenir une méthode permettant d'optimiser le choix des variables à mesurer expérimentalement", et ainsi déterminer celles qui ont la plus grande influence et dont les mesures sont accessibles. Nous pourrions ainsi réduire l'incertitude sur les prédictions du modèle, en optimisant la calibration.


