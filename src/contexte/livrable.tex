\subsection{Objectifs poursuivis}

Avant d'expliquer en détails les objectifs poursuivis par le projet,
nous proposons au lecteur d'avoir une vision globale sur ces objectifs.
Pour ce faire, une \emph{fiche objectif} comme celle présentée 
lors de la conférence sur 
la méthode du Ludion\footnote{Conférence du 5 novembre 2015 
sur la méthode de Ludion par Paul-Hubert des \textsc{Mesnards}.} 
semble parfaitement adaptée.
Nous y avons par conséquent accordé une importance fondamentale
lors de la rédaction de cette section.
Celle-ci se trouve à la Figure~\ref{fig:fiche_obj} 
dans l'Annexe~\ref{ann:fiche_obj}.


On va travailler sur le Modèle LNAS, adapté à la modélisation du blé. 
Le client a déja implémenté ce modèle en C++. 
Nous allons pour notre part l'implémenter sous \textsc{Julia}.

Ce modèle était d'abord utilisé pour modéliser la betterave, 
plus simple à modéliser que le blé. 
En effet, pour la betterave il suffit de modéliser 
ses racines (partie utile) et ses feuilles.
On va modéliser le blé, on rajoute la tige et les grains (partie utile du blé)
pour maximiser le rendement des cultures en minimisant l’apport d’intrans.
Nous allons donc utiliser la plateforme Pygmalion 
pour implémenter ce modèle du blé,
en nous basant sur un document fourni par le client.

Il y a environ 30 paramètres dans le modèle. Une fois le modèle implémenté, nous essaierons donc d'estimer ces paramètres par calibration, grâce aux outils fournis par la statistique.

De plus, les mesures étant difficiles à réaliser et ayant un coût, on ne peut pas mesurer toutes les variables tout le temps. A terme, le but serait \"d'obtenir une méthode permettant d'optimiser le choix des variables à mesurer expérimentalement\", celles qui ont la plus grande influence et dont les mesures sont accessibles. Nous pourrions ainsi réduire l'incertitude sur les prédictions du modèle, en optimisant la calibration.

Le modèle LNAS est un modèle de Markov caché : 
\begin{itemize}
  \item $n$  temps (discret)     
  \item $X_n$ variable au temps $n$       
  \item $Q\sub{grain}, Q\sub{leaf}... $ biomasse des grains, feuilles...      
  \item $f$ fonction qui permet de passer du temps $n$ au temps $n+1$     
  \item $P$ paramètres      
  \item En précipations, température... au temps $n$
\end{itemize}

\begin{equation}
  X_{n+1} = f(X_n,E_n,P,n)
\end{equation} 

$Q\sub{prod}(n)$ proportionnelle à $(1-\exp{(-\lambda \, \text{LAI}(n))})$ 
avec LAI la quantité lumineuse reçue par la plante

Nous allons travailler sur le Modèle LNAS, adapté à la modélisation du blé. 
Le client a déjà implémenté ce modèle en C++. 
Nous allons pour notre part l'implémenter sous Julia.

Il y a environ 30 paramètres dans le modèle. Une fois le modèle implémenté, nous essaierons donc d'estimer ces paramètres par calibration, grâce aux outils fournis par la statistique.

De plus, les mesures étant difficiles à réaliser et ayant un coût, on ne peut pas mesurer toutes les variables tout le temps. A terme, le but serait "d'obtenir une méthode permettant d'optimiser le choix des variables à mesurer expérimentalement", et ainsi déterminer celles qui ont la plus grande influence et dont les mesures sont accessibles. Nous pourrions ainsi réduire l'incertitude sur les prédictions du modèle, en optimisant la calibration.

On déduit $Q\sub{grain}(n)$... grâce à $X_n$ et la fonction d'allocation de biomasse.


