\subsection{Objectifs poursuivis}
Avant d'expliquer en détails les objectifs poursuivis par le projet,
nous proposons au lecteur d'avoir une vision globale sur ces objectifs.
Pour ce faire, une \emph{fiche objectif} comme celle présentée 
lors de la conférence sur 
la méthode du Ludion\footnote{Conférence du 5 novembre 2015 
sur la méthode de Ludion par Paul-Hubert des \textsc{Mesnards}.} 
semble parfaitement adaptée.
Nous y avons par conséquent accordé une importance fondamentale
lors de la rédaction de cette section.
Celle-ci se trouve à la Figure~\ref{fig:fiche_obj} 
dans l'Annexe~\ref{ann:fiche_obj}.
On trouvera égalemment à la Figure~\ref{fig:arbre_perti}
dans l'Annexe~\ref{ann:arbre_perti}, l'arbre de pertinence
qui nous a permis de rappelé le but poursuivi ainsi
que de décomposer le projet en un ensemble de sous-objectifs.

Notre objectif est de travailler sur le modèle LNAS
adapté à la modélisation du blé et plus particulièrement de l'implémenter
sur la plateforme PyGMAlion-Julia du laboratoire Digiplante.

% Pour ce faire, il nous semble nécessaire de décrire les étapes
% qui vont nous permettre de réaliser cette tâche le plus efficacement possible.

Il s'agit donc en premier lieu de comprendre comment les modèles
sont représentés sur la plateforme. 
On rappelera notamment qu'en informatique tout est discrétisé
que ce soit implicitement ou explicitement.
À titre d'exemple, calculer une intégrale est un problème
qui nécessite d'intégrer sur un domaine qui est théoriquement continu,
l'ordinateur cependant utilisera toujours des méthodes qui discrétise
l'aire sous une courbe.
L'exemple explicite semble une façon adéquate d'abordée
la façon dont un modèle est considéré sous PyGMAlion.
En effet, il s'agit de construire un système dynamique discret,
c'est-à-dire où le temps est découpé en un nombre fini d'intervalles $t_n$.
Le système est alors représenté sous la forme
\[
  X_{n+1} = f_n(X_n, U_n, P)
\]
où $X_n$ correspond aux variables d'état du système (la production
de biomasse) au temps $n$, $P$ les paramètres (qui dépendent des types
de plantes en question) et $U_n$ qui représente les variables
environnementales telles que la quantité de radiation, température,
teneur en eau du sol.
Une fois ces 3 groupes identifiés, il s'agit de déterminer la fonction $f_n$ qui relie les variables d'état entre elles.
La dernière étape de modélisation consiste à déterminer
les fonctions qui régissent la répartition de biomasse 
à travers les différents organes de la plante.
Ceci dépendra évidemment du type de plante considéré
(le cas de la betterave est par exemple considérablement
plus simple que celui du blé).

Il y a environ 30 paramètres dans le modèle. Une fois le modèle implémenté, nous essaierons donc d'estimer ces paramètres par calibration,
grâce à des outils statistiques.

De plus, les mesures expérimentales étant difficiles à réaliser 
et ayant un coût, on ne peut pas mesurer toutes les variables tout le temps. 
À terme, le but serait \enquote{d'obtenir une méthode permettant d'optimiser
le choix des variables à mesurer 
expérimentalement}\footnote{Fiche de présentation du projet enjeu.
\textsc{Digiplante}, 2015.},
et ainsi déterminer celles qui ont la plus grande influence 
et dont les mesures sont accessibles. 
Nous pourrions ainsi réduire l'incertitude sur les prédictions du modèle,
en optimisant la calibration.






