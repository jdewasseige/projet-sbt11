\subsection{Organisation et partage des tâches}
Il nous reste un dernier point à décrire, celui de la \emph{communication}
au sein du groupe.
Nous utilisons Slack\footnote{\url{https://slack.com/}} qui est un logiciel
de plus en plus utilisé pour les travaux de groupe ainsi que dans les start-ups.
Il permet d'éviter de devoir alterner entre plusieurs applications comme les mails,
DropBox et Twitter, puisqu'il permet d'être connecté
à celles-ci au sein de l'application.
On peut également créer plusieurs \emph{channels} pour séparer la communication
entre les différentes tâches.
Par exemple dans ce projet nous avons les \emph{channels} suivantes :
\texttt{general}, \texttt{etude-documentaire}, 
\texttt{planning} et \texttt{dev\_plate-forme}.
On trouve aussi un système d'historique et de gestion de fichiers efficace.

L'ensemble des tâches ainsi que leur répartition pendant l'année
est associé au planning \textsc{Gantt} qui se trouve
dans l'Annexe~\ref{ann:planning} à la page~\pageref{ann:planning}.
