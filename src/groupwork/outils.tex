\subsection{Outils de travail collaboratif}
Pour des raisons pratiques et esthétiques, nous avons décider d'écrire
nos rapports en \LaTeX{}.
Il s'agissait donc de trouver la meilleure façon de partager le code source
et de pouvoir contrôler les changements apportés au document.
Une première idée pourrait être d'utiliser ShareLaTeX qui propose une plate-forme
de compilation en ligne ainsi qu'un système de gestion de versions
assez simple à utiliser.
Nous n'avons pas choisi cette solution notamment pour les raisons suivantes.
L'utilisateur doit être connecté dès qu'il veut travailler sur le projet,
le système de compilation est assez lent et l'utilisateur n'est pas libre
d'utiliser son éditeur de texte ou son visualisateur de \textsc{pdf} favori.

Pour palier aux problèmes décrits ci-dessus, le logiciel \texttt{git}
associé à GitHub est une très bonne alternative.
Il permet en effet à chaque membre du groupe de travailler sans être connecté
ainsi que d'utiliser son éditeur et compilateur favori.
Chaque membre travaille donc de son côté en faisant des \emph{commits}
et lorsqu'il juge que son travail est utile pour les autres, 
il \emph{push} sur le serveur.
L'algorithme du fusion, \emph{merge}, permet également de fusionner intelligemment
les lignes d'un fichier qui ont été modifiées par plusieurs membres.
Le dernier point à souligner est que \texttt{git} permet une gestion des branches,
particulièrement pratique lorsqu'on veut développer une partie du projet
sans risquer de créer des erreurs dans le programme principal.

Nous combinons donc ces deux outils pour 
\begin{enumerate}
  \item implémenter le modèle LNAS blé dans la plateforme
  Pygmalion en \textsc{Julia} (une description plus détaillée de
  l'objectif attendu pour cette partie est décrite dans la
  section~\ref{sec:livrable} à la page~\pageref{sec:livrable})
  pour le client dont le code source est sur la plateforme GitLab,
  \item rédiger l'étude documentaire en partageant le code \LaTeX{}
  à l'aide d'un dossier sur 
  GitHub\footnote{\url{https://github.com/jdewasseige/projet-sbt11}}.
\end{enumerate}
