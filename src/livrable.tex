\section{Livrable}
\subsection{Implémentation du modèle LNAS sous Julia}
On va travailler sur le Modèle LNAS, adapté à la modélisation du blé. 
Le client a déja implémenté ce modèle en C++. 
Nous allons pour notre part l'implémenter sous Julia.

Ce modèle était d'abord utilisé pour modéliser la betterave, 
plus simple à modéliser que le blé. 
En effet, pour la betterave il suffit de modéliser 
ses racines (partie \"utile\") et ses feuilles.
On va modéliser le blé, on rajoute la tige et les grains (partie "utile" du blé)
pour maximiser le rendement des cultures en minimisant l’apport d’intrans.
Nous allons donc utiliser la plateforme Pygmalion pour implémenter ce modèle du blé,
en nous basant sur un document fourni par le client.

Il y a environ 30 paramètres dans le modèle. Une fois le modèle implémenté, 
nous allons donc procéder à une analyse de paramètres et de leur influence. 
En effet, on ne peut pas mesurer tous les paramètres tout 
le temps (cela ayant un coût) 
Le but sera d'obtenir une liste \emph{des paramètres pertinents}.

Le modèle LNAS est un modèle de Markov caché : 
\begin{itemize}
  \item $n$  temps (discret)     
  \item $X_n$ variable au temps $n$       
  \item $Q\sub{gain}, Q\sub{leaf}... $ biomasse des grains, feuilles...      
  \item $f$ fonction qui permet de passer du temps $n$ au temps $n+1$     
  \item $P$ paramètres      
  \item En précipations, température... au temps $n$
\end{itemize}

\begin{equation}
  X_{n+1} = f(X_n,E_n,P,n)
\end{equation} 

$Q\sub{prod}(n)$ proportionnelle à $(1-\exp{(-\lambda \, \text{LAI}(n))})$ 
avec LAI la quantité lumineuse reçue par la plante

C’est un modèle de Markov caché, il n'y a pas de mémoire du passé.
\[
  P(X_{n+1}/X_0,…,X_n) = P(X_{n+1} / X_n) 
\] 

On déduit $Q\sub{grain}(n)$... grâce à $X_n$ et la fonction d'allocation de biomasse.
