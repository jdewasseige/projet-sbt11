\section{Résumé}

Dans le cadre de  notre projet enjeu, ce rapport retrace notre étude bibliographique. 
Ce projet porte sur l'étude des modèles de croissance des plantes, avec pour objectif la mise en oeuvre d'une méthode permettant de déterminer les paramètres d'un modèle par calibration, ainsi que les mesures les plus pertinentes à réaliser pour avoir la meilleure précision possible.
A l'heure actuelle, le client nous a donné un premier objectif, qui concerne l'implémentation d'un modèle, le modèle LNAS en Julia.
 

Après nous être intéressé au fonctionnement biologique général des plantes et à l'histoire de la modélisation des plantes, nous avons étudié différents modèles de modélisation de la croissance des plantes à l'aide des documents fournis par le client.
Nous avons ainsi pu appréhender les enjeux du projet, et avoir une vision générale de la modélisation des plantes.

Nous avons également jeté les bases de notre mode d'organisation et de fonctionnement en équipe afin d'être le plus productif possible dans notre projet.