\section{Résumé}

Ce rapport retrace notre étude bibliographique en vue de réaliser notre projet enjeu. L'objectif général de ce projet étant de travailler sur des modèles de croissance des plantes qui peuvent être utilisé dans l'agriculture de précision avec un premier objectif donné par le client d'implémentation du modèle LNAS de croissance du blé sous Julia. 

Après s'être intéressé  à des généralités biologiques sur le fonctionnement des plantes et sur l'histoire de la modélisation des plantes, nous avons à l'aide des documents fournis par le client et de ressources, notamment en littérature mathématique, étudié plus précisément les modèles proches du modèle GreenLab de modélisation des plantes, qui est un modèle de Markov Caché en temps discret, en particulier le modèle LNAS qui en est une version simplifiée. Nous avons ainsi à la fois pu appréhender les enjeux du projet, s'être constitué une vision globale de "l'état de l'art" et des pratiques courantes en modélisation de croissance des plantes, et commencé à travailler sur de premiers modèles (LNAS betterave et LNAS blé). 

Nous avons également jeté les bases de notre mode d'organisation et de fonctionnement en équipe pour pouvoir être pleinement fonctionnels dans nos travaux.