\section{Résumé}
Ce rapport a été réalisé dans le cadre du projet enjeu Santé
et Biotechnologies à CentraleSupélec.
Le but premier de ce projet est de se retrouver face
à un problème ouvert où il n'existe pas forcément de solution exacte
comme nous sommes habitués à avoir dans nos autres cours.
Un autre but poursuivi est de se retrouver en situation concrète
de collobaration avec un client extérieur,
ce qui change complètement du cadre habituel des travaux de groupes.
Finalement, ce projet nous a appris à travailler en autonomie ainsi
qu'à mener des recherches selon une démarche scientifique.

Notre projet enjeu se concentre sur
le \emph{développement d'outils mathématiques 
pour l'agriculture de précision} avec l'équipe Digiplante.
La première partie de ce titre est assez compréhensible,
la deuxième cependant suscite des nombreuses questions qui méritent
des réponses.
La croissance exponentielle de la population au cours des dernières
années ont obligé la science à avancer dans le domaine de l'agriculture.
Le contexte de réchauffement climatique oblige également 
à suivre une démarche scientifique qui se veut respectueuse 
de l'environnement.

Dans un premier temps, nous exposons une \emph{synthèse documentaire}
balayant le plus largement possible le sujet de l'agriculture de précison,
synthèse qui est le fruit de notre étude bibliographique.
Les références utilisées se trouvent en annexe.
Au cours de cette synthèse, nous commençons par poser les bases
nécessaires, en introduisant le concept
d'agriculture de précision, des notions de physiologie des plantes
ainsi qu'un bref historique de la compréhension des plantes.
Nous entrons ensuite dans le vif du sujet en détaillant quelques modèles
et leurs caractéristiques mathématiques.

On retrouve ensuite une partie sur le contexte
et les objectifs détaillés du projet.
Les différents outils qui nous ont aidé dans la réalisation de
cette partie se trouvent en annexe.

Nous finirons par expliquer l'organisation interne du groupe
ainsi que les moyens mis en \oe{}uvre pour assurer
le bon déroulement du projet.
