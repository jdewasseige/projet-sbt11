\section{Modèle LNAS}
Ci-dessous est un exemple de blé en appliquant le modèle LNAS.
\[ {Q_o^{(n+1)}} = (1-\beta_o^{(n)}-\gamma_o^{(n)} )(Q_o)^{(n)} +\alpha_o^{(n)}q^{(n)} \]
Pour des feuilles jaunes, comme ils étaient les sénilités des feuilles verts donc ils ne reçoivent pas de biomasses qui viennent de Q, la production de biomasse. Dans ce cas, 
\[ {Q_y^{(n+1)}}=(1-\gamma_y^{(n)})Q_y^{(n)}+Y_l^{(n)}Q_l^{(n)} \]
Le temps thermique du jour n+1 peut s'écrire:
\[ {\tau}^{(n+1)}=\tau^{(n)}+max[0,T_-^{(n)}-T_c] \]
Supposant que la répartiotion de biomasse ne commence qu'après passer le temps caractéristique $\tau_g$ et  nous pouvons paramétrer la fonction $\alpha$ avec la loi log-normal:
\[ {\alpha_g^{(n)}}=F_{\log N(\mu_s,\sigma_g)}(\tau^{(n)}-\tau_g)=\frac{1}{2} (1+erf[\frac{1}{\sigma_g \sqrt{2}}\log (\frac{\tau^{(n)}-\tau_g}{t_{1/2}-\tau_g})]) \]
De même pour la fonction $\beta $ et $\gamma$, on a
\[ {\beta_o}=\eta_o F_{\log N(\mu_l, \sigma_l)}(\tau-\tau_l) \]
\[ {\gamma_l}=(1-\eta_l)F_{\log N(\mu_l, \sigma_l)}(\tau-\tau_l) \]
\[ {\gamma_y}=F_{\log N(\mu_y, \sigma_y)}(\tau-\tau_y) \]

\[ {q^{(n)}}=RUE min[SSI^{(n)}, TSI_\uparrow^{(n)}]PAR^{(n)}(1-e^{-\lambda LAI^{(n)}})+\sum_o \beta_o^{(n)}Q_o^{(n)} \]
\begin{itemize}
\item $\mathrm{Q}(n)$ : Production de biomasse au jour n par unité de surface.
\item $\o=$ $\left\lbrace grain, stem,root,greenleaf\right\rbrace$ :différent partie de blé
\item $\alpha$  :fonction de répartition
\item $\beta_o$  :fonction de reutilisation
\item $\gamma$ : fonction de sénilité
\item $q^{(n)}$: production de biomasse par photosynthèse au jour n 
\item $\mu_g=\log (\tau_{1/2}-\tau_g)$
\item $\sigma_g$: variance de lq distribution
\item $\tau_{1/2}$ : temps thermique où $\alpha_g$ vaut 1/2
\item $erf(x)$: fonction d'erreur
\item $\eta_o$: fraction de biomasse 
\item $RUE$ :efficience de conversion
\item $\mathrm{PAR}(t)$ : quantité de radiations photosynthétiquement actives par unité de surface
\item $LAI^{(n)}$ : facteur d'apres loi de Beer-Lambert 
\item $SSI^{(n)}$: index de pression stomatique 
\item $TSI^{(n)}$: index de pression thermique
\end{itemize}
L'équation d'équilibre de l'eau
\[ {R^{(n+1)}}=R^{(n)}+W^{(n)}-Es^{(n)}-Tp{(n)}-d{(n)}\]
L'eau dans la terre dépend du profondeur z,
\[ {R^{(n)}}=\int_0^{z_M}dz \theta^{(n)}(z) \]
L'évaporation et transpiration sont exprimées comme
\[ {Espot^{(n)}}=K_sET0^{(n)}e^{-\lambda LAI^{(n)}}\]
\[ {Tppot^{(n)}}=K_cET0^{(n)}(1-e^{-\lambda LAI^{(n)}})\]
\begin{itemize}
\item $R^{(n)}$: l'eau dans la terre du jour n
\item $W^{(n)}$ : l'eau d'irrigations et  précipitations au jour n
\item $Es^{(n)}$: l'eau perdu par évaporation vanant de la terre
\item $Tp$: transpiration
\item $d$: fonction qui compte le fait la saturation d'eau dans la terre
\end{itemize}
