Ce modèle était d'abord utilisé pour modéliser la betterave, 
plus simple à modéliser que le blé. 
En effet, pour la betterave il suffit de modéliser 
ses racines (partie utile) et ses feuilles.
On va modéliser le blé, on rajoute la tige et les grains (partie utile du blé)
pour maximiser le rendement des cultures en minimisant l’apport d’intrants.
Nous allons donc utiliser la plateforme Pygmalion pour implémenter ce modèle du blé,
en nous basant sur un document fourni par le client.

Le modèle LNAS est un modèle de Markov caché : 
\begin{itemize}
  \item $n$  temps (discret)     
  \item $X_n$ variable au temps $n$       
  \item $Q\sub{grain}, Q\sub{leaf}... $ biomasse des grains, feuilles...      
  \item $f$ fonction qui permet de passer du temps $n$ au temps $n+1$     
  \item $P$ paramètres      
  \item En précipations, température... au temps $n$
\end{itemize}

\begin{equation}
  X_{n+1} = f(X_n,E_n,P,n)
\end{equation} 

$Q\sub{prod}(n)$ proportionnelle à $(1-\exp{(-\lambda \, \text{LAI}(n))})$ 
avec LAI la quantité lumineuse reçue par la plante

C’est un modèle de Markov caché, il n'y a pas de mémoire du passé.
\[
  P(X_{n+1}/X_0,…,X_n) = P(X_{n+1} / X_n) 
\] 

On déduit $Q\sub{grain}(n)$... grâce à $X_n$ et la fonction d'allocation de biomasse.



\begin{itemize}
\item $\mathrm{Q}(n)$ : Production de biomasse au jour n par unité de surface.
\item $\alpha$  :fonction de répartition
\item $\beta_o$  :fonction de remobilisation
\item $\gamma$ : fonction de sénescence
\item $q^{(n)}$: production de biomasse au jour n 
\item $\mu_g=\log (\tau_{1/2}-\tau_g)$ : espérance de la distribution
\item $\sigma_g$: variance de la distribution
\item $\tau_{1/2}$ : temps thermique où $\alpha_g$ vaut 1/2
\item $erf(x)$: fonction d'erreur
\item $\eta_o$: fraction de biomasse 
\item $RUE$ : efficience de l'utilisation lumineuse
\item $\mathrm{PAR}(t)$ : quantité de radiations photosynthétiquement actives par unité de surface
\item $LAI^{(n)} = S^{(n)}_l/S = \rho_l Q^{(n)}_l $ : Leaf Area Index pour la loi de Beer-Lambert 
\item $\lambda = \sin(\psi)$ avec $\psi$ l'angle moyen des feuilles avec lumière incidente. 
\item $SSI^{(n)}$: indice de stress stomatique (partie hydrique)
\item $TSI^{(n)}$: indice de stress thermique
\end{itemize}
L'équation d'équilibre de l'eau
\[ {R^{(n+1)}}=R^{(n)}+W^{(n)}-Es^{(n)}-Tp{(n)}-d{(n)}\]
L'eau dans la terre dépend de la profondeur z,
\[ {R^{(n)}}=\int_0^{z_M}dz \theta^{(n)}(z) \]
L'évaporation et la transpiration demandées par la plante sont exprimées comme
\[ {Espot^{(n)}}=K_sET0^{(n)}e^{-\lambda LAI^{(n)}}\]
\[ {Tppot^{(n)}}=K_cET0^{(n)}(1-e^{-\lambda LAI^{(n)}})\]

L'eau effectivement transpirée ou évaporée sera le minimum  entre l'évaporation/transpiration requise et l'évaporation/transpiration maximale qui sont calculées selon les propriétés du sol et de la plante (non détaillé dans ce document).

\begin{itemize}
\item $R^{(n)}$: l'eau dans la terre du jour n
\item $W^{(n)}$ : l'eau issue de l'irrigation et de  précipitations au jour n
\item $Es^{(n)}$: l'eau perdue par évaporation depuis la terre
\item $Tp$: transpiration
\item $d$: fonction qui prend en compte la saturation d'eau dans la terre
\end{itemize}