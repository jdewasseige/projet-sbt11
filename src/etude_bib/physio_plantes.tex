\subsection{Généralités}
PARLER DES MERISTEMES ?
Tout d'abord, présentons succintement comment se développe et fonctionne une plante. 
Sans doute que la particularité la plus intéressante des plantes, et qui justifie le mieux leur étude est leur capacité à synthétiser de la matière organique (des glucides...) à partir de \ce{CO2}, d'eau et d'énergie solaire. C'est la célèbre photosynthèse qui permet ainsi à la plante de \textit{transformer de la matière minérale en matière organique}, qu'on peut présenter selon cette équation : 
\[
	\ce{6CO2 + 6H2O + \text{énergie solaire} -> C6H12O6 + O2 }
\]
Tous les éléments de la plante participent à ce processus de photosynthèse :
\begin{itemize}
	\item les racines puisent dans le sol l'eau et les sels minéraux nécessaires		
	\item les feuilles captent l'énergie solaire et le dioxyde de carbone 
  (grâce aux cellules chlorophiliennes 
  et aux stomates\footnote{Orifice de petite taille situé sur les feuilles 
  qui permet les échanges gazeux entre l'air et la plante.})
\end{itemize}

Les sucres ainsi formés apportent l'énergie nécessaire au fonctionnement de la plante et assurent son développement en permettant la synthèse de cellulose qui est l'élément consitutif principal des plantes.
On identifie ainsi les éléments qui agissent sur la croissance de la plante : 
\begin{itemize}
	\item la lumière
	\item l'eau
	\item le dioxyde de carbone
	\item la température : car la température agit sur l'ouverture des stomates et donc sur le flux des échanges gazeux
	\item l'azote, qui permet à la plante de construire les acides aminés nécessaires à l'élaboration des protéines
	\item d'autres minéraux, comme le potassium qui favorise les transferts au sein de la plante
\end{itemize}

Tous les organes de la plante s'unissent donc pour aboutir à la production de biomasse. Cette biomasse est ensuite distribuée aux organes pour permettre leur développement.