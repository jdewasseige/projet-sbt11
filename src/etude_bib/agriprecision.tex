\subsection{Concepts l'agriculture de précision}

- apparition dans le cadre de la course au progrès des rendements agricoles
- ce que c'est : un principe de gestion de parcelles agricoles
- optimisation des rendements via modification des conditions inter- et intraparcellaires
- nécessite nouvelles technologies / apparu grâce à

- idée de préserver/optimiser l'utilisation des ressources naturelles, financières et énergétiques
- rêve : aide à la décision permettant d'aider de manière générique les agriculteurs ou encore que tout soit automatiser

- objectif : récoltes les plus bénéfiques possibles en ayant une consommation minimum d'intrants (eau, engrais)et d'énergie en tenant compte de 3 facteurs généraux
  - agronomique
  - environnemental -> tenir compte des risques pour la santé et l'environnement en pronant l'application locale précise
  - économique
  
- points positifs 
  - aider à la prise de décision puisque vue globale et connaissances des conséquences des décisions
  - l'idée est ici de pouvoir s'imaginer qu'on va cliquer sur un bouton, décider combien on veut produire et connaître ce que ça va avoir comme cout et influence sur le rendement
  - réduction l'émission d'ammoniac donc économie sur l'achat d'engrais et l'utilisation d'eau
% title = {Les émissions agricoles de particules dans l’air. État des lieux et leviers d’action.},
% author = {Agence de l'Environnement et de la Maîtrise de l'Energie},
% isbn = {978-2-35838-220-5},
% url = {http://www.chambres-agriculture.fr/fileadmin/user_upload/thematiques/Produire_durablement/Energies_biomasse/Emissions_agricoles_particules.pdf}
  - bénéfique pour la culture, les sols les nappes phréatiques

- points négatifs
  - cout importants car mise en oeuvre d'un système de positionnement et matériel agricole doit être moderne et équipé
  - matériels conçus pour gestion de grandes parcelles
  
