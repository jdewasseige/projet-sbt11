\subsection{Concepts d'agriculture de précision}
La croissance exponentielle de la population~\cite{wiki:popu_mondiale}
au cours des dernières décennies et par conséquent la nécessité de nourrir
tout le monde, ne laisse pas d'autres choix que d'optimiser l'agriculture.
Cependant le concept d'optimisation de l'agriculture peut 
paraître assez flou et général.
Une définition assez intuitive d'une agriculture optimisée
peut être d'essayer d'obtenir
les récoltes les plus bénéfiques en ayant une consommation minimum d'énergie
et d'intrants (eau, engrais\dots) en tenant compte de facteurs à la fois
agronomiques, environnementaux et économiques~\cite{wiki:agri_prec}.

C'est précisemment ce que vise l'agriculture de précision.
Globalement, on dira que \enquote{l'agriculture de précision désigne
l'ensemble des techniques culturales basées sur l'utilisation
des nouvelles technologies de mesure et de traitement de l'information
spatialisée}~\cite{jullien2005agriculture}.
Le principe général consiste à caractériser le milieu dans toute sa
variabilité spatiale et non comme un ensemble homogène.
On va donc segmenter les parcelles agricoles en sous-parcelles
que l'on considère comme homogènes vis-à-vis de certains paramètres.
À l'aide des informations récoltées sur l'ensemble des parcelles,
on va pouvoir cartographier le domaine.
Un exemple type de l'application de ce principe à un champ
est donné dans la Figure~\ref{fig:vegedrones}.

À ce jour, les coûts engendrés par les installations technologiques
nécessaires pour utiliser les systèmes d'agriculture de précision
sont très élévés
mais l'investissement est souvent rentabilisé à long terme, surtout 
dans le cas de grandes parcelles.
En effet, cette nouvelle forme d'agriculture ne permet pas seulement
d'être bénéfique pour les sols et nappes phréatiques,
elle engendre également une diminution des dépenses de l'agriculteur
par une consommation moins importante d'engrais et d'eau.

À cette diminution d'utilisation d'engrais s'associe une réduction
des émissions de particules néfastes pour l'environnement telles que des nitrates, phosphates
et pesticides~\cite{emission_agri_particules}.
Ce dernier point semble d'autant plus important dans le contexte
écologique actuel où le respect de l'environnement est devenu nécessaire
tant d'un point vue légal que moral.

\begin{figure}
  \begin{center}
    \includegraphics[scale=0.42]{./img/agridrone.jpeg}
  \end{center}
  \caption{Représentation d'une application d'agriculture de 
  précision~\cite{vegedrones}.
  On retrouve un système d'acquisition d'images sur drones ainsi
  qu'une analyse rétroactive des données récupérées.}
  \label{fig:vegedrones}
\end{figure}