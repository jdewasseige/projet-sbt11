Afin d'avoir une première approche globale du sujet, 
nous commençons par introduire
le concept d'agriculture de précision et son rôle de plus en plus important.
Il semble en effet important de commencer par expliquer \emph{pourquoi}
des laboratoires comme Digiplante existent et plus généralement \emph{pourquoi}
faire des progrès dans le domaine de l'agriculture de précision est nécessaire.

On présente ensuite des concepts généraux sur la physiologie des plantes
ainsi que la réaction de photosynthèse, critique dans la création
de biomasse.
Ceux-ci permettront par la suite de comprendre plus aisément
les modèles mathématiques utilisés pour décrire la croissance des plantes.

Une fois le contexte général posé, nous expliquons l'histoire de la modélisation 
des plantes. Ceci va nous permettre de comprendre quelles ont été les différentes 
sources d'influences tout au long de l'étude de la modélisation des plantes.

On décrira finalement le modèle LNAS appliqué au blé, ceci en se rappelant que l'objectif poursuivi est de prédire le rendement
de la biomasse totale de la plante.

