% Quelle est la problématique générale et l'état de l'art dans le domaine ?

%Pour aborder le projet de façon efficace, une étude documentaire réalisée
%de façon critique est nécessaire.
%Par critique nous entendons une analyse 
%et une réflection sur nos sources.

La méthodologie que nous avons suivi pour écrire l'analyse bibliographique
est la suivante : lire et comprendre le mieux possible plusieurs sources
d'informations pertinentes, pour ensuite faire comme si on devait l'expliquer 
soi-même à quelqu'un qui ne connait pas le sujet.
Cette démarche nous permet d'éviter de simplement faire de la retranscription
puisqu'elle nous oblige à trouver notre propre façon d'expliquer 
à partir d'informations provenant de différentes sources.
Les éléments clés sont ainsi naturellement sélectionnés.

%Afin d'être efficace dans la façon d'aborder le projet, 
%il s'agit de comprendre les sujets en rapport avec n
%de développe

Afin d'avoir une première approche globale du sujet, nous commençons par introduire
le concept d'agriculture de précision et son rôle de plus en plus important.
Il semble en effet important de commencer par expliquer \emph{pourquoi}
des laboratoires comme Digiplante existent et plus généralement \emph{pourquoi}
faire des progrès dans le domaine de l'agriculture de précision est nécessaire.

Une fois le contexte général posé, nous expliquons l'histoire de la modélisation 
des plantes. Ceci va nous permettre de comprendre quelles ont été les différentes 
sources d'influences tout au long de l'étude de la modélisation des plantes.
% état de l'art, apparition des différents domaines tq maths, biologie